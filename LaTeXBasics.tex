
	\section{Introduction}
	This is the first section.
	
	Lorem ipsum dolor sit amet. consectetuer adipiscing elit. Etiam lortisfacilisis sem. Nullam nec mi et neque pharetra sollicitudin. Praesent imperdietmi nec ante. Donex ullamcorper, felis non sodakes...
	
	\section{The Table}
	Table \ref{table:data} shows how to add a table caption and reference a table
	\begin{table}[h!]
		\centering
		\begin{tabular} {|| c c c c ||}
			\hline
			Col1 & Col2 & Col2 & Col3 \\ [0.5ex]
			\hline \hline
			1 & 6 & 87837 & 787 \\
			\hline
			2 & 7 & 78 & 5415 \\
			\hline
			3 & 545 & 778 & 7507 \\
			\hline
			4 & 545 & 18744 & 7560 \\ [1ex]
			\hline
			
		\end{tabular}
		\caption{Table to test captions and labels.}
		\label{table:data}
	\end{table}
	
	\section*{The Joke}
	\addcontentsline{toc}{section}{Unnumbered Section}
	
	Here goes some more text which does not belong in any of the previous sections. For example this is the space for the joke with the fox and the hedgehog.
	$\ldots$ HA Ha ha ha $\ldots$ Well that was awkward.
	
	\section{First Section after the unnumbered one}
	To show that one sometimes has to Press
	\emph{F5} multiple times to update the table of contents
	
	\section*{Bibliography}
	\addcontentsline{toc}{section}{Bibliography}
	\LaTeX{} \cite{calderonDetectionAnalysisFlipper2024} is a set of macros built atop \TeX{} \cite{martoyoSoftwareDefinedRadio2018} 
	\bibliographystyle{plain}
	We choose the "plain" reference style 

	Entries are in the BScReferences.bib file.
	Although the Bibliography should always be at the end of any paper as seen in \ref*{section:References}.
	
	\section{Skelett vom Exposee}
	\begin{itemize}
		\item[Wieso:] relevant  Motivation : Da learning bei doing effektiver ist als rein theoretisch kann man mit dem HackRF One Studenten mit den Problemen der Smart Home Sicherheit vertraut machen. Und dies auf eine Einsteigerfreundliche Art und Weise. 
		\emph{fehlt noch: Zitat dass learning by doing besser als theoretisch und Motivation Stuierenden überhaupt SDRs und Smart Home Sicherheit beizubringen}
		
		\item[Was:]
		Beschreibung der Aufgabe: Eine Übung erstellen welche Studierenden die Nutzung eines Software Defined Radio näherbringt. Zudem herausfinden welche forensischen, offensiven und auch Radio-technische Möglichkeiten offenstehen. Zum Beispiel wäre eine einfache Spoofing Attacke auf GSM oder GPS Geräte denkbar sowie Abhörtechniken oder einfache Signalanalyse um möglicherweise Kommunikation oder Datentransfer zu entschlüsseln und abzufangen.
		\emph{fehlt noch: Was is ein SDR (HackRF One), Wie funktionieren die verschiedenen Attacken/ Ansätze was ist der Umfang der Aufgabe}
		
		\item[Wie:]
		Herausfinden was überhaupt schon gemacht wurde im Smart Home und SDR Bereich (anhand von ) $\Rightarrow$ Herausfinden welche Ansätze und so auf dem HackRF One machbar sind  besonders auch wegen der begrenzten Zeit im Umfang einer Aufgabe  $\Rightarrow$  Aufgabenstellung und Aufgabe erarbeiten  $\Rightarrow$  Leute aus Zielgruppe testen lassen zur genaueren Beschränkung bzw. Hinzugabe von Hinweisen um Zeitaufwand nicht zu überspannen $\Rightarrow$ Aufgabe finalisieren $\Rightarrow$ Nochmal finaler Testlauf (wenn möglich andere Leute) $\Rightarrow$ Ergebnisse Dokumentieren
		\emph{fehlt noch: Literatur Ideen/ schon gefundene Paper erwähnen}
		
		\item [Wie testen:]	
		Testläufe mit Zeitmessung / Zeitlimit und unterschiedlichen Kenntnisständen, Vergleich der erwähnten Methoden mit State of The Art und auf Relevanz prüfen
		\emph{fehlt noch: Wer, Wie, Wo}
		
		\item[Wieso Vorgehensweise gut:] 	
		Wegen der Breite an Möglichkeiten und des Testens. Weil es sehr unwahrscheinlich ist dass ich einen Exploit finden würde in der kurzen Zeit $\Rightarrow$ Auf wissenschaftliche Vorarbeit zurückgreifen
		
		\item[Meilensteine:] 
		\begin{enumerate}
				\item Literaturübersicht/ Methodenübersicht 
				\item Grundimplementieren (Basisschnittstelle für alle anderen Methoden ausarbeiten) 
				\item Jede Woche oder je 2 Wochen eine Anwendung \item  Testlauf 1: 2 1/2  Monate vor Ablauf der Frist 
				\item  Testlauf	2: 1 1/2 Monate vor Enden 
				\item  Abgabe der vorläufigen Abschlussarbeit Paar Wochen vor Ende 
				\item  Abgabe Arbeit.
			\emph{fehlt noch: Kulanz bei Zeit und Verantwortung}
		\end{enumerate}
	
		
		
	\end{itemize}
	
